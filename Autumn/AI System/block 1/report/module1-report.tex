\documentclass{article}
\usepackage[T1,T2A]{fontenc}
\usepackage{listings}
\usepackage{geometry}
\geometry{
 	a4paper,
  	top=25mm, 
  	right=15mm, 
  	bottom=25mm, 
  	left=30mm
}
\lstset{
	language=Java,
}

\begin{document}
\section{Введение}
В первом блоке происходит знакомство с такими понятиями как база знаний и онтология. На их основе можно строить вопросно-ответные системы (экспертные), которые своим поведением могут напоминать современные LLM. Примером таких систем являются чатботы на сайтах банков, управляющих компаний etc. Для ознакомления я тем, на чем такие програмы базируются, были сделаны лабораторные работы 1 и 2.
\section{Анализ требований}
\subsection{Лабораторная работа №1, часть 1}
Основными требованиями к базе знаний были характеристики содержащихся в ней фактов и правил, а именно:
\begin{itemize}
\item факты с 1 аргументом
\item факты с 2 аргументами
\item правила, содержащие логические операции (И/ИЛИ etc.)
\end{itemize}
На основе этих трех структур можно строить самые примитивные вопросно-ответные системы, способные отвечать на заданные вопросы да/нет или предоставлять список сущностей, подходящих под заранее определенный набор фильтров.
\subsection{Лабораторная работа №1, часть 2}
В данной лабораторной работе требовалось перевести ранее созданную базу знаний в онтологию Protege. Онтологии отличатся от баз знаний устройством и процессом создания. Так объекты БЗ - классы онтологии, отношения БЗ - свойства онтологии. Соответственно, главным требованием является корректность переноса и способность выполнить те же запросы, что и в БЗ с тем же вариантом.
\subsection{Лабораторная работа №2}
Во второй лабораторной работе как раз предлогается создать систему на основе БЗ/онтологии - систему поддержки принятия решений, которая на основе данных из запроса будет выводить данные с рекомендациями. Соответсвенно главным требованием является корректный парсинг запроса и составление корректных запросов к БЗ/онтологии. 

\section{Изучение основных концепций и инструментов}
\subsection{Prolog}
Пролог является логическим языком программирования. Основные концепции языка представлены ниже. 
\begin{itemize}
\item Декларативный стиль программирования -- следовательно программист описывает что должно быть сдлеано, а не как.
\item Факты -- описывают истины о некторой теме. Например персонаж(иван) - значит иван является персонажем.
\item Правила -- логические зависимости между фактами. Напрмиер горожанин(Х) :- персонаж(Х) - значит любой X горожанин если он персонаж.
\item Унификация -- процесс сопоставления запроса с содержимым БЗ с целью найти данные, которые соотвествуют запросу. 
\item  Обратный поиск -- в отличии от простых реализаций поиска решений, к которым может придти самостоятельно, Prolog использует механизм обратного поиска, который позволяет ему возвращаться к чекпоинтам в случае неудачного поиска и начинать с конкретного места, а не с начала, что позволяет эффективно обходить все возможные варианты решений в поисках нужного.
\end{itemize}

\subsection{Protégé}
Данная программа используется для создания онтологий - позволяет моделировать и организовывать знания.
Основныt концепции ниже.
\begin{itemize}
\item Онтология -- формальное описание знаний в определенной предметной области. Она включает в себя классы (понятия), свойства (отношения между понятиями) и индивиды (конкретные объекты).
\item Классы и подклассы -- представляют категории или типы объектов, а подклассы — это специфические категории, которые наследуют свойства от родительских классов.
\item Свойства (отношения) -- описывают связи между классами и индивидами. Свойства могут быть объектными (связь между двумя объектами) или датальными (связь объекта с данными, такими как числа или строки).
\item Индивиды -- конкретные экземпляры классов.
\item Reasoners --  возможность использовать механизмы вывода (reasoners), такие как Hermit или Pellet, для автоматической проверки логической целостности онтологии. Эти reasoners могут делать выводы на основе определенных фактов и правил, обнаруживать логические ошибки или находить скрытые связи между понятиями.
\end{itemize}
И Protégé и Prolog могут быть использованы для построения экспертных систем, которые в ограниченой области могут довольно эффктивно находить ответы на заданные пользователем вопросы. Однако до современных LLM(больние языковые модели), если говорить о взаимдоействии в формате чата, они не дотягивают ввиду узконаправленности БЗ/онтологий.
\subsection{Библиотека JPL}
В рамках лабораторных я ознакомился с библиотекой JPL для Java, которая позволяет взаимодействовать с базами знаний Prolog из java-программы. Основной командой для взаимодействия является:
\begin{lstlisting}
Query facts = new Query(
                "consult",
                new Term[]{new Atom("path to *.pl file")}
        );
Query q = new Query("fact or rule", new Term[]{X})
q.allSolutions()
\end{lstlisting}
Через Query происходит подключение БЗ, обращение к БЗ и извлечение ответов.
\section{Реализация системы искусственного интеллекта (системы поддержки принятия решений)}
Для реализации СППР необходимо создать парсер, который будет вынимать из строки запроса параметры, на основе которых будет формироваться список обращений к БЗ. Так же необходимо создать метод, который будет обращаться к БЗ и записывать ее ответы. Это позволит создать простейшую вопросно-ответную систему.
\section{Оценка и интерпретация результатов}
В результате лабораторных работ первого блока была создана система помощи прнятия решений соответствующая требованиям, были изучены алгоритмы создания баз знаний и онтологий. Для дальнейшего улучшения разработанной системы можно расширить базу знаний, добавиви в нее больше фактов, а так же расширить количество шаблонов, которые принимает программа, чтобы общение с системой казалось более "натуральным".
\section{Заключение}
Системы помощи принятия решений довольно часто используются чтобы облегчить работу сотрудников службы поддержки или автоматизировать проверку гипотез, моделировать сценарии и проверять логическую корректность своих предположений.



\end{document}